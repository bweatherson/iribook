\documentclass[fleqn]{beamer}
\let\Tiny=\tiny

 \setbeamertemplate{navigation symbols}{} 

% \usetheme{Madrid}
 \usetheme[numbering=none, progressbar=foot]{metropolis}
 \usecolortheme{wolverine}
 \usepackage{color}
 \usepackage{MnSymbol}
% \usepackage{movie15}


 \usepackage{tabulary}

\AtBeginSection[]
{
   \begin{frame}
	\Huge{\color{darkblue} \insertsection}
   \end{frame}
}

\usepackage{color}
\definecolor{darkgreen}{rgb}{0,0.7,0}
\definecolor{darkblue}{rgb}{0,0,0.8}
\definecolor{darkred}{rgb}{0.6,0,0}


\def\latexmode{beamer}
\def\mytitle{Interests and Evidence}
\def\myauthor{Brian Weatherson}
\def\mydate{February 26, 2017}
\title{\mytitle}
\author{\myauthor}
\institute[Michigan/Arché]{University of Michigan, Ann Arbor and Arché, University of St Andrews}
\date[]{\mydate}

\begin{document}

\frame{\titlepage}


\section{Encroachment, Reduction and Explanation}
\label{encroachmentreductionandexplanation}

\begin{frame}

\frametitle{Red-Blue Game}
\label{red-bluegame}

Rules of the game:

\begin{enumerate}
\item Two sentences will be written on the board, one in red, one in blue.

\item You get two choices.

\item First, you pick a colour, red or blue.

\item Second, you say whether the sentence in that colour is true or false.

\item If you are right, you win. If not, you lose.

\item Let's imagine that if you win, you get \$50, and if you lose, you get nothing.

\end{enumerate}

Assume that you know the rules of the game, and nothing else relevant.

\end{frame}

\begin{frame}

\frametitle{Red-Blue Game}
\label{red-bluegame}

An instance of the Red-Blue Game 

\begin{itemize}
\item \textcolor{darkred}{Two plus two equals four.}

\item \textcolor{darkblue}{\textit{Knowledge and Lotteries} was published before \textit{Knowledge and Practical Interests}.}

\end{itemize}

Intuitions:

\begin{itemize}
\item In ordinary circumstances, I know the blue sentence is true.

\item The only rational play is Red-True.

\end{itemize}

\end{frame}

\begin{frame}

\frametitle{Knowledge Norm of Action}
\label{knowledgenormofaction}

\begin{itemize}
\item Pragmatic encroachment theories can easily explain this; I lose knowledge about publication dates when playing the game.

\item Non-pragmatic theories have a harder time explaining it.

\item Saying that knowledge isn't sufficient for action won't help here; then it isn't clear why knowledge of the rules licences Red-True as a rational play.

\item Saying I never knew the fact about publication dates won't help; the example structure generalises widely.

\end{itemize}

\end{frame}

\begin{frame}

\frametitle{Odds not Stakes}
\label{oddsnotstakes}

\begin{itemize}
\item This is a low stakes situation - it's just \$50.

\item But it is a long odds bet.

\item More precisely, Blue-True is rational only if it is at least as probable that Blue is true as that 2+2=4.

\item And that probability claim isn't very plausible.

\end{itemize}

\end{frame}

\begin{frame}

\frametitle{The Conditional Principle}
\label{theconditionalprinciple}

I endorse these principles as constraints on knowledge:

\begin{itemize}
\item If the agent knows that \emph{p}, then for any question they have an interest in, the answer to that question is identical to the answer to that question conditional on \emph{p}.

\item When an agent is considering the choice between two options, the question of which option has a higher expected utility given their evidence is a question they have an interest in.

\end{itemize}

\end{frame}

\begin{frame}

\frametitle{Reduction and Explanation}
\label{reductionandexplanation}

\begin{itemize}
\item Those principles are meant to not just be extensionally adequate.

\item They are meant to explain why agents lose knowledge when considering some sets of options, like in the Red-Blue game.

\item In some sense, they are meant to be part of reductive explanations.

\end{itemize}

\end{frame}

\begin{frame}

\frametitle{Inputs to the Explanation}
\label{inputstotheexplanation}

These reductive explanations take as primitive inputs

\begin{itemize}
\item Evidential Probability

\item Evidence

\end{itemize}

I'm not going to worry about evidential probability here, but I am going to worry a lot about evidence.

\end{frame}

\section{The Problems with Evidence}
\label{theproblemswithevidence}

\begin{frame}

\frametitle{The Red-Blue Game and Evidence}
\label{thered-bluegameandevidence}

Consider a version of the game where

\begin{itemize}
\item The red sentence is two plus two equals four.

\item The blue sentence is something that, if known, would be part of the agent's evidence.

\end{itemize}

Hypothesis:

\begin{itemize}
\item We can get situations where the only rational play is Red-True, but in ordinary circumstances, the agent would know the blue sentence is true.

\end{itemize}

\end{frame}

\begin{frame}

\frametitle{An Example}
\label{anexample}

\begin{itemize}
\item I see someone, call them Rahul, across the room in a restaurant in Ann Arbor.

\item Rahul is someone I know well, and can recognise, but I had no idea he was in town.

\item Still, the ordinary situation is that I know Rahul is here.

\item Indeed, the ordinary situation is that Rahul being in this restaurant is part of my evidence.

\end{itemize}

Now play a version of the game with:

\begin{itemize}
\item \textcolor{darkred}{Two plus two equals four.}

\item \textcolor{darkblue}{Rahul is in this restaurant.}

\end{itemize}

\end{frame}

\begin{frame}

\frametitle{The Challenge}
\label{thechallenge}

\begin{itemize}
\item This doesn't threaten the extensional adequacy of the conditional principle.

\item This set of views is consistent: E=K, and I don't know Rahul is here, so it's not part of my evidence that Rahul is here, so the evidential probability of Rahul being in Ann Arbor is not high enough to choose Blue.

\item But this explanation is not a reductive explanation of why I don't know Rahul is here.

\item It reasons from the lack to knowledge to the lack of evidence, and I want an explanation that goes the other way around.

\end{itemize}

\end{frame}

\begin{frame}

\frametitle{Some Ways Out}
\label{somewaysout}

\begin{enumerate}
\item Insist that evidence is only ever phenomenological, and the red-blue game never defeats phenomenological knowledge.

\item Give up on the project of providing reductive explanations for why changing practical circumstances lead to loss of knowledge.

\end{enumerate}

Neither seems particularly plausible.

\end{frame}

\begin{frame}

\frametitle{Multiple Solutions}
\label{multiplesolutions}

One cost of the explanation being non-reductive is that the following position is also consistent:

\begin{itemize}
\item E=K

\item Agents loses knowledge that \emph{p} when the evidential probability of \emph{p} is not close enough to one.

\item Since \emph{p} is part of my evidence, its evidential probability is 1, so it is close enough to 1.

\item So there is no threat from pragmatic encroachment to knowledge here.

\end{itemize}

A non-reductive account of when pragmatic effects matter is, in this case, a non-predictive account.

\end{frame}

\section{Gamifying the Problem}
\label{gamifyingtheproblem}

\begin{frame}

\frametitle{Newcomb's Problem as a Game}
\label{newcombsproblemasagame}

\begin{itemize}
\item It is interesting to think of some philosophical problems as games, especially when they involve interactions of rational agents.

\item Here, for example, is the game table for Newcomb's problem, with the familiar human as Row, and the demon as Column.

\end{itemize}


\begin{center}
\begin{tabular}{r | c c}
 & Predict 1 Box & Predict 2 Boxes \\ \hline
Choose 1 Box & 1000, 1 & 0,0 \\
Choose 2 Boxes & 1001, 0 & 1, 1
\end{tabular}
\end{center}
\pause


Note that the unique Nash equilibrium of the game is the bottom right corner.

\end{frame}

\begin{frame}

\frametitle{The Interpretation Game}
\label{theinterpretationgame}

There are two players:

\begin{enumerate}
\item Human

\item Radical Interpreter

\end{enumerate}

Here are their goals:

\begin{itemize}
\item Radical interpreter assigns mental states (including evidence) to human in such a way as to correctly predict human's actions (assuming human is rational).

\item Human acts so as to maximise evidential expected utility, where the evidence is what the radical interpreter says the evidence is.

\end{itemize}

\end{frame}

\begin{frame}

\frametitle{A Version of the Game}
\label{aversionofthegame}

\begin{itemize}
\item Human faces a choice between taking and declining a bet on \emph{p}.

\item If bet wins, it wins 1 util, if it loses, it loses 100 utils.

\item \emph{p} is like the claim that Rahul is in the restaurant; it is unclear whether it is in human's evidence.

\item If $K$ is the rest of human's evidence, then $\Pr(p | K) = 0.9$.

\item Radical interpreter has to choose whether \emph{p} is part of the evidence or not.

\item Human has to decide whether to take the bet or not.

\item Radical interpreter gets what they want if human takes the bet iff \emph{p} is part of their evidence.

\end{itemize}

\end{frame}

\begin{frame}

\frametitle{Table for the Game}
\label{tableforthegame}


\begin{center}
\begin{tabular}{r | c c}
& $p \in E$ & $p \notin E$ \\ \hline
Take the bet & 1, 1 & -9.1, 0 \\
Decline the bet & 0, 0 & 0, 1
\end{tabular}
\end{center}


\begin{itemize}
\item Since the bet is rational iff \emph{p} is part of evidence, radical interpreter wins in the top-left and lower-right quadrants, and loses otherwise.

\item In the bottom row, human gets a payout of 0, since the bet is declined.

\item In the top-right, the bet is a sure winner, so it's expected return is 1.

\item In the top-left, bet wins with probability 0.9, so its expected payout is --9.1.

\end{itemize}

\end{frame}

\begin{frame}

\frametitle{Equilibria of the Game}
\label{equilibriaofthegame}

There are two Nash equilibria for the game - I've bolded them below.


\begin{center}
\begin{tabular}{r | c c}
& $p \in E$ & $p \notin E$ \\ \hline
Take the bet & \textbf{1, 1} & -9.1, 0 \\
Decline the bet & 0, 0 & \textbf{0, 1}
\end{tabular}
\end{center}


That corresponds to the conditional principle not setting a unique solution to what the agent's evidence\slash knowledge is.

\end{frame}

\begin{frame}

\frametitle{First Attempt}
\label{firstattempt}

\begin{itemize}
\item Rational play in the Interpretation Game involves playing one part of a Nash equilibrium strategy.

\item Any play, by either human or radical interpreter, is potentially part of a Nash equilibrium strategy.

\item So it is indeterminate whether \emph{p} is part of agent's evidence or not.

\item That's not a terrible solution - it is reductive and interest-relative, but let's see if we can do better.

\end{itemize}

\end{frame}

\section{A Better Approach}
\label{abetterapproach}

\begin{frame}

\frametitle{Stag Hunt}
\label{staghunt}


\begin{center}
\begin{tabular}{r | c c}
& $a$ & $b$  \\\hline
$A$ & 5, 5 & 0, 4 \\
$B$ & 4, 0 & 2, 2
\end{tabular}
\end{center}


\begin{itemize}
\item This game has two equilibria, $Aa$ and $Bb$.

\item Let's talk about the choice between them.

\end{itemize}

\end{frame}

\begin{frame}

\frametitle{Pareto-Dominant}
\label{pareto-dominant}

\begin{itemize}
\item The $Aa$ equilibrium is better for both players than the $Bb$ equilibrium.

\item That is, it is \textbf{Pareto-dominant}.

\item Some theorists think we should select Pareto-dominant equilibria, when they are available.

\end{itemize}

\end{frame}

\begin{frame}

\frametitle{Risk-Dominant}
\label{risk-dominant}

\begin{itemize}
\item Each player does best playing $Bb$ if they think it is 50\slash 50 which equilibrium strategy the other player will play.

\item That is (simplifying a little), the $Bb$ strategy is \textbf{risk-dominant}.

\item Some other theorists think we should select risk-dominant equilibria, when they are available.

\end{itemize}

\end{frame}

\begin{frame}

\frametitle{Dominance Principles}
\label{dominanceprinciples}

I think risk-dominance is a more sensible equilibrium choice rule, since Pareto-dominance can lead to selecting weakly dominated strategies, as here.


\begin{center}
\begin{tabular}{r | c c}
& $a$ & $b$ \\ \hline
$A$ & 4, 4 & 0, 4 \\
$B$ & 4, 0 & 2, 2
\end{tabular}
\end{center}


\end{frame}

\begin{frame}

\frametitle{Solving the Interpretation Game}
\label{solvingtheinterpretationgame}

\begin{itemize}
\item The equilibria are equally preferred by the radical interpreter.

\item But if human is making a choice while being 50\slash 50 on what the radical interpreter will pick, they maximise expected utility by declining the bet.

\item So the equilibrium: Decline bet, $p \notin E$ is risk-dominant.

\end{itemize}

\end{frame}

\begin{frame}

\frametitle{My Theory}
\label{mytheory}

Evidence human has is the evidence radical interpreter says they have, assuming both players choose a risk-dominant equilibrium.

\begin{itemize}
\item This theory is reductive; it doesn't presuppose what human's evidence is before we say what they know.

\item And it is interest-relative.

\end{itemize}

\end{frame}

\section{Problems with the Theory}
\label{problemswiththetheory}

\begin{frame}

\frametitle{Problem One: Too Complex}
\label{problemone:toocomplex}

The conditional principle had two theoretical motivations:

\begin{enumerate}
\item Always maximise evidential expected utility

\item Conditionalizing on what you know doesn't change any relevant question

\end{enumerate}

No game-theoretic story is going to be as plausible, or as simple.

\end{frame}

\begin{frame}

\frametitle{Problem Two: Possibility of Evidence Matters}
\label{problemtwo:possibilityofevidencematters}

Change one variable in the interpretation game:

\begin{itemize}
\item Now the downside to losing the bet is 15 utils, not 100.

\item That gives us the following game table, whose risk-dominant equilibria is the top left corner.

\end{itemize}


\begin{center}
\begin{tabular}{r | c c}
& $p \in E$ & $p \notin E$ \\ \hline
Take the bet & 1, 1 & -0.6, 0 \\
Decline the bet & 0, 0 & 0, 1
\end{tabular}
\end{center}


\end{frame}

\begin{frame}

\frametitle{Problem Two: Possibility of Evidence Matters}
\label{problemtwo:possibilityofevidencematters}

\begin{itemize}
\item This is odd.

\item It isn't odd because the stakes matter; that's just interest-relativity.

\item It is odd because normally, if $\Pr(p) = 0.9$, and the agent was facing a bet on $p$ at 15--1 odds, we'd say their practical interests block knowledge that $p$.

\item It turns out that if $p$ is evidence if known, the probabilistic threshold for it being known is higher than if it is certainly not evidence.

\item That's surprising, though I'm not sure if there are clear intuitions here.

\end{itemize}

\end{frame}

\section{Conclusion}
\label{conclusion}

\begin{frame}

\frametitle{Summary}
\label{summary}

\begin{enumerate}
\item Evidence is what a radical interpreter, who thought you were rational and wanted to predict your behaviour, would think it is.

\item This will be interest-sensitive in any number of unclear cases.

\item How the radical interpreter solves this problem depends on much more contentious aspects of game theory than orthodox utility theory, but any viable solution will be interest-relative in some sense.

\end{enumerate}

\end{frame}

\mode<all>
\input{talk-beamer-footer}

\end{document}\mode*
