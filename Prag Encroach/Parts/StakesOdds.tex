\subsection{Stakes, Odds and Experiments}

In a so far unpublished note, Mark Schroeder \citeyearpar{SchroederStakes} has argued that interest-relative invariantists have erred by stressing variation in stakes as being relevant to knowledge. He argues, using examples of forced choice, that what is really relevant is the odds at which the agent has to make bets. Of course due to the declining marginal utility of material goods, high stakes bets will often be long odds bets. So there's a correlation between stakes and odds. But when the correlation comes apart, Schroeder argues convincingly that it's the odds and not the stakes that are relevant to knowledge.

The variable threshold view of belief I've been defending agrees with Schroeder's judgments. Interests affect belief because  whether someone believes \(p\) depends \textit{inter alia} on whether their credence in \(p\) is high enough that any bet on \(p\) they actually face is a good bet. Raising the stakes of any bet on \(p\) does not change that, but changing the odds of the bets on \(p\) they face does change it. And that explains why agents don't have knowledge, or even justified belief, in some of the examples that motivate other interest-relative invariantists.

Although I think the variable threshold view gets those cases right, I don't take those examples to be a crucial part of the argument for the view. The core argument is that the view provides a better answer to Sturgeon's challenge of how we should integrate credences and beliefs into a single model. If it turned out that the facts about the examples were less clear than we thought, that wouldn't \textit{undermine} the argument for the variable threshold view, since those facts weren't part of the original argument. But if it turned out that the facts about those examples were quite different to what the variable threshold view predicts, that may \textit{rebut} the view, since it would then be shown to make false predictions.

This kind of rebuttal may be suggested by various recent experimental results, such as the results in \cite{May2010} and \cite{FeltzZarpentine2010}. I'm going to concentrate on the latter set of results here, though I think that what I say will generalise to related experimental work. Feltz and Zarpentine gave subjects related vignettes, such as the following pair. (Each subject only received one of the pair.)

\begin{description}
\item[High Stakes Bridge] John is driving a truck along a dirt road in a caravan of trucks. He comes across what looks like a rickety wooden bridge over a yawning thousand foot drop. He radios ahead to find out whether other trucks have made it safely over. He is told that all 15 trucks in the caravan made it over without a problem. John reasons that if they made it over, he will make it over as well. So, he thinks to himself, `I know that my truck will make it across the bridge.'

\item[Low Stakes Bridge] John is driving a truck along a dirt road in a caravan of trucks. He comes across what looks like a rickety wooden bridge over a three foot ditch. He radios ahead to find out whether other trucks have made it safely over. He is told that all 15 trucks in the caravan made it over without a problem. John reasons that if they made it over, he will make it over as well. So, he thinks to himself, `I know that my truck will make it across the bridge.' \citep[??]{FeltzZarpentine2010}
\end{description}

\noindent Subjects were asked to evaluate John's thought. And the result was that 27\% of the participants said that John does not know that the truck will make it across in \textbf{Low Stakes Bridge}, while 36\% said he did not know this in \textbf{High Stakes Bridge}. Feltz and Zarpentine say that these results should be bad for interest-relativity views. But it is hard to see just why this is so.

Note that the change in the judgments between the cases goes in the direction that the variable threshold view predicts. The change isn't trivial, even if due to the smallish sample size it isn't statistically significant in this sample. But should the variable threshold view have predicted a larger change? To figure this out, we need to ask three questions.

\begin{enumerate*}
\item What are the costs of the bridge collapsing in the two cases?
\item What are the costs of not taking the bet, i.e., not driving across the bridge?
\item What is the rational credence to have in the bridge's sturdiness given the evidence John has?
\end{enumerate*}

None of these are specified in the story given to subjects, so we have to guess a little as to what the subjects' views would be. 

Feltz and Zarpentine say that the costs in``High Stakes Bridge [are] very costly---certain death---whereas the costs in Low Stakes Bridge are likely some minor injuries and embarrassment.'' \cite[??]{FeltzZarpentine2010} I suspect both of those claims are wrong, or at least not universally believed. A lot more people survive bridge collapses than you may expect, even collapses from a great height.\footnote{In the West Gate bridge collapse in Melbourne in 1971, a large number of the victims were underneath the bridge; the people on top of the bridge had a non-trivial chance of survival. That bridge was 200 feet above the water, not 1000, but I'm not sure the extra height would matter greatly. Again from a slightly lower height, over 90\% of people on the bridge survived the I-35W collapse in Minneapolis in 2007.} And once the road below a truck collapses, all sorts of things can go wrong, even if the next bit of ground is only 3 feet away. (For instance, if the bridge collapses unevenly, the truck could roll, and the driver would probably suffer more than minor injuries.)

We aren't given any information as to the costs of not crossing the bridge. But given that 15 other trucks, with less evidence than John, have decided to cross the bridge, it seems plausible to think they are substantial. If there was an easy way to avoid the bridge, presumably the \textit{first} truck would have taken it.

But the big issue is the third question. John has a lot of information that the bridge will support his truck. If I've tested something for sturdiness two or three times, and it has worked, I won't even think about testing it again. Consider what evidence you need before you'll happily stand on a particular chair to reach something in the kitchen, or put a heavy television on a stand. Supporting a weight is the kind of thing that either fails the first time, or works fairly reliably. Obviously there could be some strain-induced effects that cause a subsequent failure\footnote{As I believe was the case in the I-35W collapse.}, but John really has a lot of evidence that the bridge will support him.

Given those three answers, it seems to me that it is a reasonable bet to cross the bridge. At the very least, it's no more of an unreasonable bet than the bet I make every day crossing a busy highway by foot. So I'm not surprised that 64\% of the subjects agreed that John knew the bridge would hold him. At the very least, that result is perfectly consistent with the variable threshold view, if we make plausible assumptions about how the subjects would answer the three numbered questions above.

And as I've stressed, these experiments are only a problem for the variable threshold view if the subjects are reliable. I can think of two reasons why they might not be. First, subjects tend to massively discount the costs and likelihoods of traffic related injuries. In most of the country, the risk of death or serious injury through motor vehicle accident is much higher than the risk of death or serious injury through some kind of crime or other attack, yet most people do much less to prevent vehicles harming them than they do to prevent criminals or other attackers harming them.\footnote{See the massive drop in the numbers of students walking or biking to school, reported in \cite{Ham2008}, for a sense of how big an issue this is.} Second, only 73\% of this subjects in \textit{this very experiment} said that John knows the bridge will support him in \textbf{Low Stakes Bridge}. This is just absurd. Unless the subjects endorse an implausible kind of scepticism, something has gone wrong with the experimental design. Given the fact that the experiment points broadly in the direction of the theory I favour, and that with some plausible assumptions it is perfectly consistent with that theory, and the unreliability of the subjects, I don't think this kind of experimental work threatens the variable threshold view.