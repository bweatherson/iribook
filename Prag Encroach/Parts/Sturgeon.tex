


Traditional epistemology deals with beliefs and their justification. Bayesian epistemology deals with degrees of belief and their justification. In some sense they are both talking about the same thing, namely epistemic justification. Two questions naturally arise. Do we really have two subject matters here (degrees of belief and belief \textit{tout court}) or two descriptions of the one subject matter? If just one subject matter, what relationship is there between the two modes of description of this subject matter?

The answer to the first question is I think rather easy. There is no evidence to believe that the mind contains two representational systems, one to represent things as being probable or improbable and the other to represent things as being true or false. The mind probably does contain a vast plurality of representational systems, but they don't divide up the doxastic duties this way. If there are distinct visual and auditory representational systems, they don't divide up duties between degrees of belief and belief \textit{tout court}, for example. If there were two distinct systems, then we should imagine that they could vary independently, at least as much as is allowed by constitutive rationality. But such variation is hard to fathom. So I'll infer that the one representational system accounts for our credences and our categorical beliefs. (It follows from this that the question \cite{Bovens1999} ask, namely what beliefs \textit{should} an agent have given her degrees of belief, doesn't have a non-trivial answer. If fixing the degrees of belief in an environment fixes all her doxastic attitudes, as I think it does, then there is no further question of what she should believe given these are her degrees of belief.) 
