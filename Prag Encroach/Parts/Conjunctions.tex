Blome-Tillman also argues that Interest Relative Invariantism is committed to certain kinds of counterintuitive conjunctions. The form of the conjunctions are that two people differ in what they know, although the two are alike in some respects relevant to knowledge. The short version of what I'm going to argue is that these conjunctions are always acceptable unless we have good reason to believe the two are alike in \textit{every} respect relevant to knowledge. But before we get to the examples, I want to note some points about `enough'. Let's start with an example unconnected to knowledge.

\noindent George and Ringo both have \$6000 in their bank accounts. They both are thinking about buying a new computer, which would cost \$2000. Both of them also have rent due tomorrow, and they won't get any more money before then. George lives in New York, so his rent is \$5000. Ringo lives in Syracuse, so his rent is \$1000. Clearly, (1) and (2) are true.

\numbex{0}{
\item Ringo has enough money to buy the computer.
\item Ringo can afford the computer.
}

\noindent And (3) is true as well, though there's at least a reading of (4) where it is false.

\numbex{2}{
\item George has enough money to buy the computer.
\item George can afford the computer.
}

\noindent Focus for now on (3). It is a bad idea for George to buy the computer; he won't be able to pay his rent. But he has enough money to do so; the computer costs \$2000, and he has \$6000 in the bank. So (3) is true. Admittedly there are things close to (3) that aren't true. He hasn't got enough money to buy the computer and pay his rent. You might say that he hasn't got enough money to buy the computer given his other financial obligations. But none of this undermines (3). 

The point of this little story is to respond to an argument Blome-Tillmann \citeyearpar[??]{MBT2009} makes concerning knowledge and the idea of having enough evidence to know. Here is how he puts the argument. (I've changed the numbering and some terminology for consistency with this chapter.)

\begin{quote}
\noindent Suppose that John and Paul have exactly the same evidence, while John is in a low-stakes situation towards \(p\) and Paul in a high-stakes situation towards \(p\). Bearing in mind that Interest Relative Invariantism is the view that whether one knows \(p\) depends on one's practical situation, Interest Relative Invariantism entails that one can truly assert:

\numbex{4}{
\item John and Paul have exactly the same evidence for \(p\), but only John has enough evidence to know \(p\), Paul doesn't.
}
\end{quote}

\noindent And this is meant to be a problem, because (5) is intuitively false.

But Interest Relative Invariantism doesn't entail any such thing. Paul does have enough evidence to know that \(p\), just like George has enough money to buy the computer. Paul can't know that \(p\), just like George can't buy the computer, because of their practical situations. But that doesn't mean he doesn't have enough evidence to know it. So, contra Blome-Tillmann, Interest Relative Invariantism doesn't entail this problematic conjunction.

In a footnote attached to this, Blome-Tillmann tries to reformulate the argument.

\begin{quote}
\noindent I take it that having enough evidence to `know \(p\)' in \(C\) just means having evidence such that one is in a position to `know \(p\)' in \(C\), rather than having evidence such that one `knows \(p\)'. Thus, another way to formulate (5) would be as follows: `John and Paul have exactly the same evidence for \(p\), but only John is in a position to know \(p\), Paul isn't.' \cite[??]{MBT2009}
\end{quote}

\noindent The `reformulation' is obviously bad as a reformulation, since having enough evidence to know \(p\) isn't the same as being in a position to know it. To see this, note that  having enough money to buy the computer is not the same as being in a position to buy it. George has enough money to buy the computer, but he isn't in a position to buy it.

But set that aside. If the original argument was bad, perhaps this isn't simply a reformulation of a bad argument, but a different and better argument against Interest Relative Invariantism. The argument would be that Interest Relative Invariantism entails (6), which is false.

\numbex{5}{
\item John and Paul have exactly the same evidence for \(p\), but only John is in a position to know \(p\), Paul isn't.
}

\noindent This objection, however, fails on the simple ground that (6) is true. Or, at least, there is no reason to believe that (6) is false. The argument Blome-Tillman makes seems simply to be an appeal to the unintuitiveness of (6). But that appeal loses its force if we have good theoretical reasons to think that sentences like (6) can be true. And indeed we have such good theoretical reasons, reasons that are independent of Interest Relative Invariantism.

Any epistemological theory that denies that what one is in a position to know supervenes on one's evidence will allow that sentences like (6) can be true.\footnote{The supervenience claim here might seem insanely strong if we take it to be a \textit{modal} supervenience claim. But since John and Paul are in the same world, presumably that's more than Blome-Tillman is really committed to. He's only committed to the claim that any two agents \textit{in the same world} that have the same evidence have the same knowledge. So we can't argue against the presupposed supervenience claim by considering cases where one person knows that \(p\) while a modal counterpart of theirs has the same evidence although \(p\) is false.} In particular, any epistemological theory that allows for the existence of defeaters which do not supervene on the possession of evidence will imply that conjunctions like (6) can be true. Now I think any particular claim about the existence of a defeater that doesn't supervene on evidence will be controversial. But there are so many different kinds of candidates that it should be obvious that there are some such candidates. Here are three possibilities; I hope that each reader finds at least one persuasive. 

\begin{quote}
\noindent \textit{Logic and the Oracle} \\ Graham, Crispin and Ringo have an audience with the Delphic Oracle, and they are told \(\neg p \vee q\) and \(\neg \neg p\). Graham is a relevant logician, so if he inferred \(p \wedge q\) from these pronouncements, his belief in the invalidity of disjunctive syllogism would be a doxastic defeater, and the inference would not constitute knowledge. Crispin is an intuitionist logician, so if he inferred \(p \wedge q\) from these pronouncements his belief in the invalidity of double negation elimination would be a doxastic defeater, and the inference would not constitute knowledge. Ringo has no deep views on the nature of logic. Moreover, in the world of the story classical logic is correct. So if Ringo were to infer \(p \wedge q\) from these pronouncements, his belief would constitute knowledge. Now Graham's and Crispin's false beliefs about entailment are not \(p \wedge q\)-relevant evidence, so all three of them have the same \(p \wedge q\)-relevant evidence, but only Ringo is in a position to know \(p \wedge q\). \smallskip

\noindent \textit{Missing iPhone} (after \cite{Harman1973}). \\ Lou and Andy both get evidence \(E\), which is strong inductive evidence for \(p\). If Lou were to infer \(p\) from \(E\), his belief would constitute knowledge. Andy has just missed a phone call from a trusted friend. The friend left a voicemail saying \(\neg p\), but Andy hasn't heard this yet. If Andy were to infer \(p\) from \(E\), his friend's phone call and voicemail would constitute defeaters, so he wouldn't know \(p\). But phone calls and voicemails you haven't got aren't evidence you have. So Lou and Andy have the same \(p\)-relevant evidence, but only Lou is in a position to know \(p\). 

\smallskip \noindent \textit{Fake Barns and Motorcycles} (after \cite{Gendler2005}) \\ Bob and Levon are travelling through Fake Barn Country. Bob is on a motorcycle, Levon is on foot. They are in an area where the barns are, surprisingly, real for a little ways around. On his motorcycle, Bob will soon come to fake barns, but Levon won't hit any fakes for a long time. They are both looking at the same real barn. If Bob inferred it was a real barn, not a fake, the fakes he is speeding towards would be defeaters. But Levon couldn't walk that far, so those barns don't defeat him. So Bob and Levon have the same evidence, but only Levon is in a position to know that the barn is real.
\end{quote}

\noindent I'm actually not sure what plausible theory would imply that what different agents are in a position to know depends on \textit{nothing} except for what evidence they have. The only theory that I can imagine with that consequence is the conjunction of evidentialism about justification and a justified true belief theory of knowledge. So really there's no reason to think that implying sentences like (6) is a mark against a theory.

It's been suggested to me\footnote{By both Jeremy Fantl and Juan Comesa\~{n}a.} that there are other more problematic conjunctions in the neighbourhood. For instance, we might worry that Interest Relative Invariantism implies that (7) is true.

\numbex{6}{
\item John and Paul are alike in every respect relevant to knowledge of \(p\), but John is in a position to know \(p\), and Paul isn't.
}

\noindent That would be problematic, but there's no reason to think Interest Relative Invariantism implies it. Indeed, Interest Relative Invariantism entails that the first conjunct is false, since John and Paul are unlike in one respect that Interest Relative Invariantism loudly insists is relevant. Perhaps we can do better with (8).

\numbex{7}{
\item John and Paul are alike in every respect relevant to knowledge of \(p\) except their practical interests, but John is in a position to know \(p\), and Paul isn't.
}

\noindent That is something Interest Relative Invariantism implies, but it seems more than a little question-begging to use its alleged counterintuitiveness against Interest Relative Invariantism. After all, it's simply a statement of Interest Relative Invariantism itself. If someone had alleged that Interest Relative Invariantism should be accepted because it was so intuitive, I guess noting how odd (8) looks would be a response to them. But that's not the way Interest Relative Invariantism has been defended here.\footnote{And for what it's worth, I don't think it's how it is defended by others in the literature either.} I've defended it by noting what a good job it does of handling difficult puzzles to do with the role of credences in philosophy of mind and epistemology. If the outcome is a little counterintuitive, that's not too surprising. It's par for the course when solving difficult puzzles.
