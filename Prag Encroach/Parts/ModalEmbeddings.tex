Jason Stanley had argued that the fact that Interest Relative Invariantism (hereafter, IRI) has counterintuitive consequences when it comes to knowledge ascriptions in modal contexts shouldn't count too heavily against IRI, because contextualist approaches are similarly counterintuitive. In particular, he argues that the theory that `knows' is a contextually sensitive quantifier, plus the account of quantifier-domain restriction that he developed with Zolt\'{a}n Gendler Szab\'{o} \citep{Stanley2000-STAOQD}, has false implications when it is applied to knowledge ascriptions in counterfactuals. Michael \cite{MBT2009} disagrees, but I don't think he provides very good reasons for disagreeing. In fact, Stanley's argument seems to be a very good argument against Lewisian contextualism about `knows'.\footnote{The canonical source for Lewisian contextualism is \cite{Lewis1996b}, and Blome-Tillmann defends a variant in \cite{MBT2009a}.} Let's start by reviewing how we got to this point.

Often when we say \textit{All Fs are Gs}, we really mean \textit{All C Fs are Gs}, where \(C\) is a contextually specified property. So when I say \textit{Every student passed}, that utterance might express the proposition that every student \textbf{in my class} passed. Now there's a question about what happens when sentences like \textit{All Fs are Gs} are embedded in various contexts. The question arises because quantifier embeddings tend to allow for certain kinds of ambiguity. For instance, when we have a sentence like \textit{If p were true, all Fs would be G}, that could express either of the following two propositions. (We're ignoring context sensitivity for now, but we'll return to it in a second.)

\begin{itemize*}
\item If \(p\) were true, then everything that would be \(F\) would also be \(G\).
\item If \(p\) were true, then everything that's actually \(F\) would be \(G\).
\end{itemize*}

\noindent We naturally interpret (1) the first way, and (2) the second way.

\numbex{0}{
\item If I had won the last Presidential election, everyone who voted for me would regret it by now.
\item If Hilary Clinton had been the Democratic nominee in the last Presidential election, everyone who voted for Barack Obama would have voted for her.
}

\noindent Given this, you might expect that we could get a similar ambiguity with \(C\). That is, when you have a quantifier that's tacitly restricted by \(C\), you might expect that you could interpret a sentence like \textit{If p were true, all Fs would be G} in either of these two ways. (In each of these interpretations, I've left \(F\) ambiguous; it might denote the actual \(F\)s or the things that would be \(F\) if \(p\) were true. So these are just partial disambiguations.)

\begin{itemize*}
\item If \(p\) were true, then every \(F\) that would be \(C\) would also be \(G\).
\item If \(p\) were true, then every \(F\) that is actually \(C\) would be \(G\).
\end{itemize*}

\noindent Surprisingly, it's hard to get the second of these readings. Or, at least, it is hard to \textit{avoid} the availability of the first reading. Typically, if we restrict our attention to the \(C\)s, then when we embed the quantifier in the consequent of a counterfactual, the restriction is to the things that would be \(C\), not to the actual \(C\)s.\footnote{See \cite{Stanley2000-STAOQD} and \cite{Stanley2005-STAKAP} for arguments to this effect.} 

Blome-Tillmann notes that Stanley makes these observations, and interprets him as moulding them into the following argument against Lewisian contextualism.

\begin{enumerate*}
\item An utterance of \textit{If p were true, all Fs would be Gs} is interpreted as meaning \textit{If p were true, then every F that would be C would also be G}.
\item Lewisian contextualism needs an utterance of \textit{If p were true, then S would know that q} to be interpreted as meaning \textit{If p were true, then S's evidence would rule out all $\neg$q possibilities, except those that are actually being properly ignored}, i.e. it needs the contextually supplied restrictor to get its extension from the nature of the actual world.
\item So, Lewisian contextualism is false.
\end{enumerate*}

\noindent And Blome-Tillmann argues that the first premise of this argument is false. He thinks that he has examples which undermine premise 1. But I don't think his examples show any such thing. Here are the examples he gives. (I've altered the numbering for consistency with this chapter.)

\numbex{2}{
\item If there were no philosophers, then the philosophers doing research in the field of applied ethics would be missed most painfully by the public.
\item If there were no beer, everybody drinking beer on a regular basis would be much healthier.
\item If I suddenly were the only person alive, I would miss the Frege scholars most.
}

\noindent These are all sentences of (more or less) the form \textit{If p were true, Det Fs would be G}, where \(Det\) is some determiner or other, and they should all be interpreted a la our second disambiguation above. That is, they should be interpreted as quantifying over actual \(F\)s, not things that would be \(F\) if \(p\) were true. But the existence of such sentences is completely irrelevant to what's at issue in premise 1. The question isn't whether there is an ambiguity in the \(F\) position, it is whether there is an ambiguity in the \(C\) position. And nothing Blome-Tillmann raises suggests premise 1 is false. So this response doesn't work.

Even if a Lewisian contextualist were to undermine premise 1 of this argument, they wouldn't be out of the woods. That's because premise 1 is much stronger than is needed for the anti-contextualist argument Stanley actually runs. Note first that the Lewisian contextualist needs a reading of \textit{If p were true, all Fs would be G} where it means:

\begin{itemize*}
\item If \(p\) were true, every actual \(C\) that would be \(F\) would also be \(G\).
\end{itemize*}

\noindent The reason the Lewisian contextualist needs this reading is that on their story, \textit{S knows that p} means \textit{Every $\neg p$ possibility is ruled out by S's evidence}, where the \textit{every} has a contextual domain restriction, and the Lewisian focuses on the actual context. The effect in practice is that an utterance of \textit{S knows that p} is true just in case every $\neg p$ possibility that the speaker isn't properly ignoring, i.e., isn't actually properly ignoring, is ruled out by \(S\)'s evidence. Lewisian contextualism is meant to explain sceptical intuitions, so let's consider a particular sceptical intuition. Imagine a context where:

\begin{itemize*}
\item I'm engaged in sceptical doubts;
\item there is beer in the fridge
\item I've forgotten what's in the fridge; and
\item I've got normal vision, so if I check the fridge I'll see what's in it.
\end{itemize*}

\noindent In that context it seems (6) is false, since it would only be true if Cartesian doubts weren't salient.

\numbex{6}{
\item If I were to look in the fridge and ignore Cartesian doubts, then I'd know there is beer in the fridge.
}

\noindent But the only way to get that to come out false, and false for the right reasons, is to fix on which worlds we're actually ignoring (i.e., include in the quantifier domain worlds where I'm the victim of an evil demon), but look at worlds that would be ruled out with the counterfactually available evidence. We don't want the sentence to be false because I've actually forgotten what's in the fridge. And we don't want it to be true because I would be ignoring Cartesian possibilities. In the terminology above, we would need \textit{If p were true, all Fs would be Gs} to mean \textit{If p were true, then every actual C that were F would also be G}. We haven't got any reason yet to think that's even a \textit{possible} disambiguation of (6). 

But let's make things easy for the contextualist and assume that it is. Stanley's point is that the contextualist needs even more than this. They need it to be by far the  \textit{preferred} disambiguation, since in the context I describe the natural reading of (6) (given sceptical intuitions) is that it is false because my looking in the fridge wouldn't rule out Cartesian doubts. And they need it to be the preferred reading even though there are alternative readings that are (a) easier to describe, (b) of a kind more commonly found, and (c) true. Every principle of contextual disambiguation we have pushes us away from thinking this is the preferred disambiguation. This is the deeper challenge Stanley raises for contextualists, and it hasn't yet been solved.