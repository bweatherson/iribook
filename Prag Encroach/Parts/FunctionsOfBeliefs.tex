Frank \cite{Ramsey1931} provides a clear statement of one of the functional roles of credences; their connection to action. Of course, Ramsey did not take himself to be providing one-third of the functional theory of credence. He took himself to be providing a behaviourist/operationalist reduction of credences to dispositions. But we do not have to share Ramsey's metaphysics to use his key ideas. I'm going to focus on two ideas of Ramsey's here. First, the idea that it's distinctively \textit{betting} dispositions that are crucial to the account of credence. And second, the idea that all sorts of actions in everyday life constitute bets.

The first idea lives on today most prominently in the work on `representation theorems'.\footnote{See \cite{Maher1993} for the most developed account in recent times.} What a representation theorem shows is that for any agent whose pairwise preferences satisfy some structural constraints, there is a probability function and a utility function such that the agent prefers bet \(X\) to bet \(Y\) just in case the expected utility of \(X\) (given that probability and utility function) is greater than that of \(Y\). Moreover, the probability function is unique (and the utility function is unique up to positive affine transformations). Given that, it might seem plausible to identify the agent's credence with this probability function, and the agent's (relative) values with this utility function.

The functionalist goes along with much, but not quite all, of this picture. The betting preferences are an important part of the functional role of a credence; indeed, they just are the output conditions. But there are two other parts to a functional role: an input condition and a set of internal connections. So the functionalist thinks that the betting dispositions are not quite sufficient for having credences. A pre-programmed automaton might have dispositions to accept (or at least move as if accepting) various bets, but this will not be enough for credences. More relevant to our purposes, a ...

%Stuff about constitutive rationality, and how it weakens the structural constraints

%Stuff about how we need to consider quite esoteric bets

%Intuitions about belief - take as given

%Conditional utility

%Digression about conditional utility in moral cases

%Summmary - Credences support some bets; beliefs support them all

