Gillian Russell and John Doris \citeyearpar{RussellDoris2008} argue that Jason Stanley's account of knowledge leads to some implausible attributions of knowledge. Insofar as my theory agrees with Stanley's about the kinds of cases they are worried about, their objections are also objections to my theory. I'm going to argue that Russell and Doris's objections turn on principles that are \textit{prima facie} rather plausible, but which ultimately we can reject for independent reasons.\footnote{I think the objections I make here are similar in spirit to those Stanley made in a comments thread on \href{http://el-prod.baylor.edu/certain_doubts/?p=616}{Certain Doubts}, though the details are new.}

Their objection relies on variants of the kind of case Stanley uses heavily in his \citeyearpar{Stanley2005} to motivate a pragmatic constraint on knowledge. Stanley imagines a character who has evidence which would normally suffice for knowledge that \(p\), but is faced with a decision where \(A\) is both the right thing to do if \(p\) is true, and will lead to a monumental material loss if \(p\) is false. Stanley intuits, and argues, that this is enough that they cease to know that \(p\). I agree, at least as long as the gains from doing \(A\) are low enough that doing \(A\) amounts to a bet on \(p\) at insufficiently favourable odds to be reasonable in the agent's circumstance.

Russell and Doris imagine two kinds of variants on Stanley's case. In one variant the agent doesn't care about the material loss. As I'd put it, the agent's indifference to material odds shortens the odds of the bet. That's because costs and benefits of bets should be measured in something like utils, not something like dollars. As Russell and Doris put it, ``you should have reservations ... about what makes [the knowledge claim] true: not giving a damn, however enviable in other respects, should not be knowledge-making.'' \citep[??]{RussellDoris2008}. Their other variant involves an agent with so much money that the material loss is trifling to them. Again, this lowers the effective odds of the bet, so by my lights they may still know that \(p\). But this is somewhat counter-intuitive. As Russell and Doris say, ``[m]atters are now even dodgier for practical interest accounts, because \textit{money} turns out to be knowledge making.'' \citep[??]{RussellDoris2008} And this isn't just because wealth can purchase knowledge. As they say, ``money may buy the \textit{instruments} of knowledge ... but here the connection between money and knowledge seems rather too direct.'' \citep[??]{RussellDoris2008}

The first thing to note about this case is that indifference and wealth aren't really producing knowledge. What they are doing is more like defeating a defeater. Remember that the agent in question had enough evidence, and enough confidence, that they would know \(p\) were it not for the practical circumstances. I've been proposing a model where practical considerations enter debates about knowledge through two main channels: through the definition of belief, and through distinctive kinds of defeaters. It seems the second channel is particularly relevant here. And we have, somewhat surprisingly, independent evidence to think that indifference and wealth do matter to defeaters.

Consider two variants on Gilbert Harman's `dead dictator' example \citep[75]{Harman1973}. In the original example, an agent reads that the dictator has died through an actually reliable source. But there are many other news sources around, defeaters, such that if the agent read them, she would lose her belief. 

In the first variant, the agent simply does not care about politics. It's true that there are many other news sources around that are ready to mislead her about the dictator's demise. But she has no interest in looking them up, nor is she at all likely to look them up. She mostly cares about sports, and will spend most of her day reading about baseball. In this case, the misleading news sources are too distant, in a sense, to be defeaters. So she still knows the dictator has died. Her indifference towards politics doesn't generate knowledge - the original reliable report is the knowledge generator - but her indifference means that a would-be defeater doesn't gain traction.

In the second variant, the agent cares deeply about politics, and has masses of wealth at hand to ensure that she knows a lot about it. Were she to read the misleading reports that the dictator has survived, then she would simply use some of the very expensive sources she has to get more reliable reports. Again this suffices for the misleading reports not to be defeaters. Even before the rich agent exercises her wealth, the fact that her wealth gives her access to reports that will correct for misleading reports means that the misleading reports are not actually defeaters. So with her wealth she knows things she wouldn't otherwise know, even before her money goes to work. Again, her money doesn't generate knowledge - - the original reliable report is the knowledge generator - but her wealth means that a would-be defeater doesn't gain traction.

The same thing is true in Russell and Doris's examples. The agent has quite a bit of evidence that \(p\). That's why she knows \(p\). There's a potential practical defeater for \(p\). But due to either indifference or wealth, the defeater is immunised. Surprisingly perhaps, indifference and/or wealth can be the difference between knowledge and ignorance. But that's not because they can be in any interesting sense `knowledge makers', any more than I can make a bowl of soup by preventing someone from tossing it out. Rather, they can be things that block defeaters, both when the defeaters are the kind Stanley talks about, and when they are more familiar kinds of defeaters.