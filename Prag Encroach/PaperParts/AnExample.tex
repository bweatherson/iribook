\subsection{A Worked Example}
To get a feel for how this theory works in practice, it's helpful to go through a particular case, such as one that is alleged by Ram Neta to be hard for interest-relative theorists to accommodate.

\begin{quote}
Kate needs to get to Main Street by noon: her life depends upon it. She is desperately searching for Main Street when she comes to an intersection and looks up at the perpendicular street signs at that intersection. One street sign says ``State Street'' and the perpendicular street sign says ``Main Street.'' Now, it is a matter of complete indifference to Kate whether she is on State Street--nothing whatsoever depends upon it. \citep[182]{Neta2007}
\end{quote}

\noindent Let's assume for now that Kate is rational; dropping this assumption introduces mostly irrelevant complications.\footnote{Neta's own treatment of the case implies, by my lights, that Kate is irrational, since he thinks that Kate does believe that she's on Main Street on this basis. I say more about how the theory applies to irrational agents in section 4.1, and what I say there would apply equally well to Kate's case if she is irrational.} Kate will not believe she's on Main Street. She would only have that belief if she took it to be settled that she's on Main, and hence not worthy of spending further effort investigating. But presumably she won't do that. The rational thing for her to do is to get confirming (or if relevant confounding) evidence for the appearance that she's on Main. If it were settled that she was on Main, the rational thing to do would be to try to relax, and be grateful that she had found Main Street. Since she has different attitudes about what to do \textit{simpliciter} and conditional on being on Main Street, she doesn't believe she's on Main Street.

So far so good, but what about her attitude towards the proposition that she's on State Street? She has enough evidence for that proposition that her credence in it should be rather high. And no practical issues turn on whether she is on State. So she believes she is on State, right?

Not so fast! Believing that she's on State has more connections to her cognitive system than just producing actions. It's true that Kate should, and will, act exactly as if she were on State Street. But that's not enough for belief. It is only rational to believe she is on State Street if the sign she's looking at is accurate. It's possible that the sign is only partially accurate, and she is on State but not Main. But unless the sign is accurate, any belief that she's on State is undersupported. And Kate knows this. So belief that she's on State stands and falls with belief that the sign is accurate. But forming a belief that the sign is accurate, i.e., settling the question of whether the sign is accurate in the affirmative, means she should form the belief that she's on Main. And that's not something she should believe, for reasons we've gone into above.

So she shouldn't believe that she's on State, because that belief is tied up too closely to belief in a proposition that she shouldn't take to be settled. If she does believe she's on State, she would have an irrational attitude towards that proposision, and that kind of irrationality would be inconsistent with knowledge. So she doesn't know, and can't know, that she's either on State or on Main.

Neta thinks that the best way for the interest-relative theorist to handle this case is to say that the high stakes associated with the proposition that Kate is on Main Street imply that certain methods of belief formation do not produce knowledge. And he argues, plausibly, that such a restriction will lead to implausibly sceptical results. But that's not the right way for the interest-relative theorist to go. What they should say, and what I do say, is that Kate can't know she's on State Street because the only grounds for that belief is intimately connected to a proposition that, in virtue of her interests, she needs very large amounts of evidence to believe.

The take-home lesson from this is that inferential connections between propositions matter a lot on the functionalist/interest-relative theory that I'm presenting. These inferential connections can come and go quickly. Imagine that Kate sees a glimpse of a food truck that looks a lot like the cupcake truck. She thinks to herself, ``It would be great to go there and get a cupcake, but I have to run, I need to be on Main Street by noon.'' She then glances at her phone and sees a message that the cupcake truck is on Main Street. Once she gets that information the question of whether the truck she's looking at is the cupcake truck is inferentially tied to the question of whether the street she's on is Main Street. She can't form the belief that the truck is the cupcake truck without settling the question of whether she's on Main Street. And a glimpse of a food truck is not enough information to settle a life-or-death question, like the question she is facing. So she should cease believing that the truck is indeed the cupcake truck, even though the epistemic probability that the truck is the cupcake truck has not fallen, and there's no reason her credence that the truck is the cupcake truck has fallen either.

This all seems like the right thing to say about the case to me. Admittedly the case is difficult, and intuitions about it aren't of great theoretical weight. But to the extent that the intuitions are of any weight, they tell against the view that beliefs should be identified with credence above a threshold. In that example, whether Kate believes a proposition can change without her credence in the proposition changing. That's inconsistent with belief just being credence above a fixed threshold.