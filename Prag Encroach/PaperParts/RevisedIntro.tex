When you pick up a volume like this one, which describes itself as being about `knowledge ascriptions', you probably expect to find it full of papers on epistemology, broadly construed. And you'd probably expect many of those papers to concern themselves with cases where the interests of various parties (ascribers, subjects of the ascriptions, etc.) change radically, and this affects the truth values of various ascriptions. And, at least in this paper, your expectations will be clearly met.

But here's an interesting contrast. If you'd picked up a volume of papers on `belief ascriptions', you'd expect to find a radically different menu of writers and subjects. You'd expect to find a lot of concern about names and demonstratives, and about how they can be used by people not entirely certain about their denotation. More generally, you'd expect to find less epistemology, and much more mind and language. I haven't read all the companion papers to mine in this volume, but I bet you won't find much of that here.

This is perhaps unfortunate, since belief ascriptions and knowledge ascriptions raise at least some similar issues. Consider a kind of contextualism about belief ascriptions, which holds that (L) can be truly uttered in some contexts, but not in others, depending on just what aspects of Lois Lane's psychology are relevant in the conversation.\footnote{The reflections in the next few paragraphs are inspired by some comments in by Stalnaker \citeyearpar{Stalnaker2008}, though I don't want to suggest the theory I'll discuss is actually Stalnaker's.}

\begin{enumerate*}
\setcounter{enumi}{11}
\renewcommand{\labelenumi}{(\Alph{enumi})}
\item Lois Lane believes that Clark Kent is vulnerable to kryptonite.
\end{enumerate*}

\noindent We could imagine a theorist who says that whether (L) can be uttered truly depends on whether it matters to the conversation that Lois Lane might not recognise Clark Kent when he's wearing his Superman uniform. And, this theorist might continue, this isn't because `Clark Kent' is a context-sensitive expression; it is rather because `believes' is context-sensitive.  Such a theorist will also, presumably, say that whether (K) can be uttered truly is context-sensitive.

\begin{enumerate*}
\setcounter{enumi}{10}
\renewcommand{\labelenumi}{(\Alph{enumi})}
\item Lois Lane knows that Clark Kent is vulnerable to kryptonite.
\end{enumerate*}

\noindent And so, our theorist is a kind of contextualist about knowledge ascriptions. But they might agree with approximately none of the motivations for contextualism about knowledge ascriptions put forward by \cite{Cohen1988}, \cite{DeRose1995} or \cite{Lewis1996b}. Rather, they are a contextualist about knowledge ascriptions solely because they are contextualist about belief ascriptions like (L).

Call the position I've just described \textbf{doxastic contextualism} about knowledge ascriptions. It's a kind of contextualism all right; it says that (K) is context sensitive, and not merely because of the context-sensitivity of any term in the `that'-clause. But it explains the contextualism solely in terms of the contextualism of belief ascriptions. The more familiar kind of contextualism about knowledge ascriptions we'll call \textbf{non-doxastic contextualism}. Note that the way we're classifying theories, a view that holds that (K) is context-sensitive both because (L) is context-sensitive \textit{and} because Cohen \textit{et al} are correct is a version of non-doxastic contextualism. The label `non-doxastic' is being used to mean that the contextualism isn't solely doxastic, rather than as denying contextualism about belief ascriptions.

We can make the same kind of division among interest-relative invariantist, or IRI, theories of knowledge ascriptions. Any kind of IRI will say that there are sentences of the form \textit{S knows that p} whose truth depends on the interests, in some sense, of $S$. But we can divide IRI theories up the same way that we divide up contextualist theories.

\begin{description}
\item[Doxastic IRI] Knowledge ascriptions are interest-relative, but their interest-rela\-tivity traces solely to the interest-relativity of the corresponding belief ascriptions.
\item[Non-Doxastic IRI] Knowledge ascriptions are interest-relative, and their interest-relativity goes beyond the interest-relativity of the corresponding belief ascriptions.
\end{description}

\noindent Again, a theory that holds both that belief ascriptions are interest-relative, and that some of the interest-relativity of knowledge ascriptions is not explained by the interest-relativity of belief ascriptions, will count as a version of non-doxastic IRI. I'm going to defend a view from this class here.

In my \cite{Weatherson2005-WEACWD} I tried to motivate Doxastic IRI. It isn't completely trivial to map my view onto the existing views in the literature, but the idea was to renounce contextualism and all its empty promises, and endorse a position that's usually known as `strict invariantism' about these classes of statements:
\begin{itemize*}
\item $S$ is justified in having credence $x$ in $p$;
\item	If $S$ believes that $p$, she knows that $p$;
\end{itemize*}
\noindent while holding that the interests of S are relevant to the truth of statements from these classes:
\begin{itemize*}
\item $S$ believes that $p$;
\item $S$ justifiably believes that $p$;
\item $S$ knows that $p$.
\end{itemize*}

\noindent But I didn't argue for all of that. What I argued for was Doxastic IRI about ascriptions of justified belief, and I hinted that the same arguments would generalise to knowledge ascriptions. I now think those hints were mistaken, and want to defend Non-Doxastic IRI about knowledge ascriptions.\footnote{Whether Doxastic or Non-Doxastic IRI is true about justified belief ascriptions turns on some tricky questions about what to say when a subject's credences are nearly, but not exactly appropriate given her evidence. Space considerations prevent a full discussion of those cases here. Whether I can hold onto the strict invariantism about claims about justified credences depends, I now think, on whether an interest-neutral account of evidence can be given. Discussions with Tom Donaldson and Jason Stanley have left me less convinced than I was in 2005 that this is possible, but this is far too big a question to resolve here.} My change of heart has been prompted by cases like those Jason \cite{Stanley2005-STAKAP} calls `Ignorant High Stakes' cases.\footnote{I mean here the case of Coraline, to be discussed in section 3 below. Several people have remarked in conversation that Coraline doesn't look to them like a case of Ignorant High Stakes. This isn't surprising; Coraline is better described as being \textit{mistaken} than \textit{ignorant}, and she's mistaken about odds not stakes. If they're right, that probably means my argument for Non-Doxastic IRI is less like Stanley's, and hence more original, than I think it is. So I don't feel like pressing the point! But I do want to note that \textit{I} thought the Coraline example was a variation on a theme Stanley originated.} But to see why these cases matter, it will help to start with why I think some kind of IRI must be true. 

Here's the plan of attack. In section 1, I'm going to argue that knowledge plays an important role in decision theory. In particular, I'll argue (a) that it is legitimate to write something onto a decision table iff the decision maker knows it to be true, and (b) it is legitimate to leave a possible state of the world off a decision table iff the decision maker knows it not to obtain. I'll go on to argue that this, plus some very plausible extra assumptions about the rationality of certain possible choices, implies that knowledge is interest-relative. In section 2 I'll summarise and extend the argument from \cite{Weatherson2005-WEACWD} that belief is interest-relative. People who are especially interested in the epistemology rather than the theory of belief may skip this. But I think this material is important; most of the examples of interest-relative knowledge in the literature can be explained by the interest-relativity of belief. I used to think all such cases could be explained. Section 3 describes why I no longer think that. Reflections on cases like the Coraline example suggests that there are coherence constraints on knowledge that go beyond the coherence constraints on justified true belief. The scope of these constraints is, I'll argue, interest-relative. So knowledge, unlike belief or justified belief, has interest-relative defeaters. That's inconsistent with Doxastic IRI, so Doxastic IRI is false.