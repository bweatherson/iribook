\subsection{Pragmatic Defeaters}

As I mentioned at the top, Jason Stanley thinks that cases like his \textbf{Ignorant High Stakes} are important arguments for the importance of interest-relativity to the theory of knowledge. Moreover, he thinks that this shows that knowledge is interest-relative in a way that doesn't merely reflect the interest relativity of belief. I now think that cases like Coraline show that is correct. It is, I guess, a little misleading to describe Coraline as \textit{ignorant} of the stakes. But she does have a false belief about the stakes, or at least about the odds.\textit{I'll return towards the end to the question of whether stakes or odds are really crucial. It should be somewhat obvious that on my theory it is odds, not stakes, that matter, and I'll say a bit more about the benefits to framing the theory that way.} Hence she doesn't know what the odds are. Hence she is ignorant of the odds, at least in the sense of lacking knowledge.

The reason Coraline's case is a problem is that if we say she knows \(p\), and we have established that knowledge structures decision problems, then the decision table Coraline faces looks like this.

\begin{center}
\begin{tabular}{r c c}
 & \textbf{\(q\)} & \textbf{\(\neg q\)}  \\
\textbf{Take bet} & \$100 & \$1  \\
\textbf{Decline bet} & 0 & 0  \\
\end{tabular}
\end{center}

\noindent Now taking the bet dominates declining the bet. So Coraline should take the bet, if this table is correct. But she shouldn't take the bet. So the table isn't correct. So Coraline must not know \(p\).

But what could be the explanation of Coraline's lack of knowledge? It isn't that she doesn't believe \(p\). We just showed that she does. It isn't even that she lacks a \textit{justified} belief in \(p\). We just showed that she has that too. It must be that her mistaken views about \(q\) somehow \textit{defeat} her claim to knowledge that \(p\). But it's not generally true that having odd views about \(q\) defeats knowledge that \(p\). It's only because of her interests, i.e., because she faces a bet that is sensitive to both \(p\) and \(q\), that mistaken views about \(q\) defeat knowledge that \(p\). Hence her lack of knowledge here is dependent, in a crucial way, on her interests.

In slogan form, the conclusion of the previous paragraph is \textit{There are pragmatic defeaters}. The cases where these arise are admittedly rare. But sometimes an agent who has a justified true belief that \(p\) doesn't know that \(p\), because of the way \(p\) is tied up with other choices that she is currently facing.

Conceptualising the influence of interests on knowledge as coming via defeaters helps see what's mistaken about some recent criticisms of interest-relative invariantism, or IRI.\footnote{Of course, I don't think all, or even most, of the influence of interests on knowledge comes via defeaters. I think it mostly comes via the interest-relativity of belief. But the focus here is on the impact interests make on knowledge that isn't mediated by the impact they make on belief.} To see the potential problem here, imagine that there is another person, call her Dora, who has the same evidence and credences as Coraline, but who does not have the option of taking this bet. According to my version of IRI, she knows that \(p\). Michael \cite{MBT2009} suggests this is implausible. As he notes, it means we can say things like (\ref{ex:SameEv}).\footnote{Compare his example about John and Paul on page 329.}

\numbex{1}{
\item \label{ex:SameEv} Coraline and Dora have exactly the same evidence for $p$, but only Dora has enough evidence to know $p$, Coraline doesn't.
}

\noindent Blome-Tillman claims this is intuitively false, and hence IRI, which entails it, is false.

The first thing we might note here is that there's an interesting ambiguity in `enough' that makes it tricky to say whether IRI really entails that (\ref{ex:SameEv}) is true. Compare the following situation.

\begin{quote}
George and Ringo both have \$6000 in their bank accounts. They both are thinking about buying a new computer, which would cost \$2000. Both of them also have rent due tomorrow, and they won't get any more money before then. George lives in New York, so his rent is \$5000. Ringo lives in Syracuse, so his rent is \$1000.
\end{quote}

\noindent In that story, is (\ref{ex:GeorgeRingo}) true or false?

\numbex{1}{
\item \label{ex:GeorgeRingo} George and Ringo have the same amount of money in their bank accounts, but only Ringo has enough money to buy the computer, George does not.
}

I think the right thing to say is probably that (\ref{ex:GeorgeRingo}) is ambiguous, with the ambiguity turning on just how we interpet `enough'. In one sense, George does have enough money to buy the computer: he has \$6000 in his bank account, and the computer costs \$2000. So in that sense, (\ref{ex:GeorgeRingo}) is false. In another (perhaps less preferred) sense, George does not have enough money to buy the computer. It is essential that he pays his rent, and if he buys the computer he won't be able to pay the rent. In that sense, (\ref{ex:GeorgeRingo}) is true. But in that sense, it isn't surprising that (\ref{ex:GeorgeRingo}) is true, since what one has enough money to buy, in that sense, depends not just on the size of one's bank account, but on one's forthcoming debts. These debts defeat George's claim to have enough money to buy the computer, even though he has the same amount of money as Ringo in his bank account.

We can see the same kind of situation arise in distinctively epistemic puzzle cases. The following case is a variant on the `dead dictator' case developed by Gilbert \cite{Harman1973}.

\begin{quote}
George and Ringo are scientists investigating whether more than 50\% of \(F\)s are \(G\)s. They plan to run a barrage of tests on this. The first is just to collect a large random sample of \(F\)s, and see whether they are \(G\)s. To save money, they get a third party to do the data collection. The report says that over 60\% of the sampled \(F\)s are \(G\)s, and the sample is large enough that this alone is enough to ground knowledge that more than 50\% of \(F\)s are \(G\)s, and both of them rationally form the belief that more than 50\% of \(F\)s are \(G\)s. But they both have large grants, and plan to run the other tests. It will turn out that Ringo's tests confirm the fact that more than 50\% of \(F\)s are \(G\)s. But George's tests will, very misleadingly, indicate that the percentage of \(F\)s that are \(G\)s is actually somewhat under 50, and that the sample they originally used was not genuinely random.
\end{quote}

\noindent After they've got the sample, but before they run the later tests, I think that Ringo knows that more than 50\% of \(F\)s are \(G\)s, but George does not. As in Harman's case, the misleading evidence surrounding George defeats his putative knowledge. But Ringo is not, in the relevant sense, surrounded by that evidence; he was never going to collect it until after he had so much evidence that he'd know to dismiss it as misleading.

Given that, is (\ref{ex:EnoughEvidence}) true or false?

\numbex{1}{
\item \label{ex:EnoughEvidence} George and Ringo have exactly the same evidence that more than 50\% of \(F\)s are \(G\)s, but only Ringo has enough evidence to know that, George doesn't.
}

\noindent As should be expected by this stage, I think (\ref{ex:EnoughEvidence}) is ambiguous. On the most natural disambiguation, it is false. George has enough evidence to know that more than 50\% of \(F\)s are \(G\)s, just like Coraline has enough evidence to know \(p\). It's just that both of their claims to knowledge are defeated by other circumstances.

There are readings though of both (\ref{ex:SameEv}) and (\ref{ex:EnoughEvidence}) on which they are both false. But the existence of these readings should not be counterintuitive once we have remembered (or reminded our intuitions) that defeaters exist. George needs more evidence to know because his existing evidence is defeated by the misleaders that surround him. Coraline needs more evidence to know because his existing evidence is defeated by her irrational attitude towards \(q\), and attitude that matters because of her practical situation. 