\subsection{Temporal Embeddings}

Michael \cite{MBT2009} has argued that tense-shifted knowledge ascriptions can be used to show that his version of Lewisian contextualism is preferable to Interest Relative Invariantism.\footnote{Blome-Tillmann calls Interest Relative Invariantism `subject-sensitive invariantism'. This is an unfortunate moniker. The only subject-\textit{insensitive} theory of knowledge has that for any \(S, T\) \(S\) knows that \(p\) iff \(T\) knows that \(p\) The view Blome-Tillmann is targetting certainly isn't defined in opposition to \textit{this} generalisation.}. His argument uses a variant of the well-known bank cases.\footnote{See \cite{Stanley2005-STAKAP} for the versions that Blome-Tillman is building on. In the interests of space, I won't repeat them yet again here.} Let \(O\) be that the bank is open Saturday morning. If Hannah has a large debt, she is in a high-stakes situation with respect to \(O\). She had in fact incurred a large debt, but on Friday morning the creditor waived this debt. Hannah had no way of anticipating this on Thursday. She has some evidence for \(O\), but not enough for knowledge if she's in a high-stakes situation. Blome-Tillmann says that this means after Hannah discovers the debt waiver, she could say (\ref{ex:ThursdayFriday}).

\numbex{1}{
\item \label{ex:ThursdayFriday} I didn't know \(O\) on Thursday, but on Friday I did.
}

\noindent But I'm not sure why this case should be problematic for any version of Interest Relative Invariantism. As Blome-Tillmann notes, it isn't really a situation where Hannah's stakes change. She was never actually in a high stakes situation. At most her perception of her stakes change; she thought she was in a high-stakes situation, then realised that she wasn't. Blome-Tillmann argues that even this change in perceived stakes can be enough to make (\ref{ex:ThursdayFriday}) true if Interest Relative Invariantism is true. Now actually I agree that this change in perception could be enough to make (\ref{ex:ThursdayFriday}) true, but when we work through the reason that's so, we'll see that it isn't because of anything distinctive, let alone controversial, about Interest Relative Invariantism.

If Hannah is rational, then given her interests she won't be ignoring \(\neg O\) possibilities on Thursday. She'll be taking them into account in her plans. Someone who is anticipating \(\neg O\) possibilities, and making plans for them, doesn't know \(O\). That's not a distinctive claim of Interest Relative Invariantism. Any theory should say that if a person is worrying about \(\neg O\) possibilities, and planning around them, they don't know \(O\). And that's simply because knowledge requires a level of confidence that such a person simply does not show. If Hannah is rational, that will describe her on Thursday, but not on Friday. So (\ref{ex:ThursdayFriday}) is true not because Hannah's practical situation changes between Thursday and Friday, but because her psychological state changes, and psychological states are relevant to knowledge.

If the version of Interest Relative Invariantism I've been defending is correct, then this is just what we should expect. It's possible for stakes to change what the subject knows without changing what the subject believes, but the cases where this happens are rare, typically involving irrational credences in somewhat related propositions. The standard kind of way in which the agent loses knowledge when the stakes rise is that she stops \textit{believing} the target proposition.

What if Hannah is, on Thursday, irrationally ignoring \(\neg O\) possibilities, and not planning for them even though her rational self wishes she were planning for them? In that case, it seems she still believes \(O\). After all, she makes the same decisions as she would as if \(O\) were sure to be true. It's true that she doesn't satisfy the canonical \textit{input} conditions for believing \(O\), but that's consistent with believing \(O\). If functionalists didn't allow some deviation from optimal input conditions, there wouldn't be any irrational beliefs.

But it's worth remembering that if Hannah does irrationally ignore \(\neg O\) possibilities, she is being irrational with respect to \(O\). And it's very plausible that this irrationality defeats knowledge. That is, you can't be irrational with respect to a proposition and know it. Irrationality excludes knowledge. That may look a little like a platitude, so it's worth spending a little time on how it can lead to some quirky results for reasons independent of Interest Relative Invariantism.\footnote{There's a methodological point here worth stressing. Doing epistemology with imperfect agents often results in facing tough choices, where any way to describe a case feels a little counterintuitive. If we simply hew to intuitions, we risk being led astray by just focussing on the first way a puzzle case is described to us.}

So consider Bobby. Bobby has the disposition to infer \(\neg B\) from \(A \rightarrow B\) and \(\neg A\). He currently has good inductive evidence for \(q\), and infers \(q\) on that basis. But he also knows \(p \rightarrow q\) and \(\neg p\). If he notices that he has these pieces of knowledge, he'll infer \(\neg q\). This inferential disposition defeats any claim he might have to \textit{know} \(q\); the inferential disposition is a kind of doxastic defeater. Then Bobby sits down with some truth tables and talks himself out of the disposition to infer \(\neg B\) from \(A \rightarrow B\) and \(\neg A\). He now knows \(q\) although he didn't know it earlier, when he had irrational attitudes towards a web of propositions including \(q\). And that's true even though his evidence for \(q\) didn't change. I assume here that irrational inferential dispositions which the agent does not know he has, and which he does not apply, are not part of his evidence, but that shouldn't be controversial.

I think Bobby's case is just like Hannah's, at least under the assumption that Hannah simply ignores the significance of \(O\) to her practical deliberation. In both cases, defective mental states elsewhere in their cognitive architecture defeat knowledge claims. And in that kind of case, we should expect sentences like (\ref{ex:ThursdayFriday}) to be true, even if they appear counterintuitive before we've worked through the details. The crucial point is that once we work through the details, we see that somewhat distant changes in the rest of the cognitive system changes what the agent knows. So, a little counterintuitively, it can be the case that an agent knows something after a distant change in the system, but not before. That's all that happens in Hannah's case.