\documentclass[fleqn]{beamer}
\let\Tiny=\tiny

 \setbeamertemplate{navigation symbols}{} 

% \usetheme{Madrid}
 \usetheme[numbering=none, progressbar=foot]{metropolis}
 \usecolortheme{wolverine}
 \usepackage{color}
 \usepackage{MnSymbol}
% \usepackage{movie15}


 \usepackage{tabulary}

\AtBeginSection[]
{
   \begin{frame}
	\Huge{\color{darkblue} \insertsection}
   \end{frame}
}

\usepackage{color}
\definecolor{darkgreen}{rgb}{0,0.7,0}
\definecolor{darkblue}{rgb}{0,0,0.8}
\definecolor{darkred}{rgb}{0.6,0,0}


\def\latexmode{beamer}
\def\mytitle{Interests and Evidence}
\def\myauthor{Brian Weatherson}
\def\mydate{June, 2017}
\title{\mytitle}
\author{\myauthor}
\institute[Michigan/Arché]{University of Michigan, Ann Arbor and Arché, University of St Andrews}
\date[]{\mydate}

\begin{document}

\frame{\titlepage}


\section{Encroachment, Reduction and Explanation}
\label{encroachmentreductionandexplanation}

\begin{frame}

\frametitle{Red-Blue Game}
\label{red-bluegame}

Rules of the game:

\begin{enumerate}
\item Two sentences will be written on the board, one in red, one in blue.

\item You get two choices.

\item First, you pick a colour, red or blue.

\item Second, you say whether the sentence in that colour is true or false.

\item If you are right, you win. If not, you lose.

\item Let's imagine that if you win, you get \$50, and if you lose, you get nothing.

\end{enumerate}
Assume that you know the rules of the game, and nothing else relevant.

\end{frame}

\begin{frame}

\frametitle{Red-Blue Game}
\label{red-bluegame}

An instance of the Red-Blue Game 

\begin{itemize}
\item \textcolor{darkred}{Two plus two equals four.}

\item \textcolor{darkblue}{\textit{Knowledge and Lotteries} was published before \textit{Knowledge and Practical Interests}.}

\end{itemize}
Intuitions:

\begin{itemize}
\item In ordinary circumstances, I know the blue sentence is true.

\item The only rational play for me is Red-True.

\end{itemize}
\end{frame}

\begin{frame}

\frametitle{Explaining the Intuitions}
\label{explainingtheintuitions}

\begin{itemize}
\item Pragmatic encroachment theories can easily explain this; I lose knowledge about publication dates when playing the game.

\item Non-pragmatic theories have a harder time explaining it.

\end{itemize}
\end{frame}

\begin{frame}

\frametitle{The Super-Knowledge Hypothesis}
\label{thesuper-knowledgehypothesis}

\begin{itemize}
\item Pragmatic encroachment cases are cases where knowledge is preserved, but knowledge doesn't suffice for action.

\item These are cases where super-knowledge is required.

\end{itemize}
\end{frame}

\begin{frame}

\frametitle{Three Objections}
\label{threeobjections}

\begin{enumerate}
\item If super-knowledge is required, then not clear why Red-True is so well motivated.\pause

\item Not clear player has any different attitude towards \emph{Red-True will win \$50} and \emph{Blue-True will win \$50}, since player doesn't super-know the rules of the game.\pause

\item This ends up treating like cases not alike.

\end{enumerate}
\end{frame}

\begin{frame}

\frametitle{The Red-Purple-Green game}
\label{thered-purple-greengame}

A sentence is written in Red on the board. Player has three options.

\begin{enumerate}
\item Say Red, then say the truth value of the sentence, and get \$50 if correct, nothing otherwise.

\item Say Purple, then get \$50.

\item Say Green, then get nothing.

\end{enumerate}
The sentence is 2+2=4. Player knows what the sentence is, the rules, and nothing else. 

\end{frame}

\begin{frame}

\frametitle{Comparing the Games}
\label{comparingthegames}

\begin{itemize}
\item It is always OK to play Purple in the Red-Purple-Green game.

\item If player knows the blue sentence is true, then an instance of the Red-Blue game just is an instance of the Red-Purple-Green game.

\item So in any Red-Blue game where it isn't OK to play Blue-True, the player doesn't know the blue sentence is true.\pause

\item This comparative argument doesn't make any knowledge-action links.

\end{itemize}
\end{frame}

\begin{frame}

\frametitle{Pragmatic Encroachment, or Scepticism}
\label{pragmaticencroachmentorscepticism}

\begin{itemize}
\item Could just deny I know the claim about publication dates.

\item But the game generates really widely.

\item Any blue sentence you couldn't bet on, when placed alongside 2+2=4, you don't know.

\item That's a really sceptical conclusion.

\end{itemize}
\end{frame}

\begin{frame}

\frametitle{Odds not Stakes}
\label{oddsnotstakes}

\begin{itemize}
\item This is a low stakes situation - it's just \$50.

\item But it is a long odds bet.

\item More precisely, Blue-True is rational only if it is at least as probable that Blue is true as that 2+2=4.

\item And that probability claim isn't very plausible.

\end{itemize}
\end{frame}

\begin{frame}

\frametitle{The Conditional Principle}
\label{theconditionalprinciple}

I endorse these principles as constraints on knowledge:

\begin{itemize}
\item If the agent knows that \emph{p}, then for any question they have an interest in, the answer to that question is identical to the answer to that question conditional on \emph{p}.

\item When an agent is considering the choice between two options, the question of which option has a higher expected utility given their evidence is a question they have an interest in.

\end{itemize}
\end{frame}

\begin{frame}

\frametitle{Reduction and Explanation}
\label{reductionandexplanation}

\begin{itemize}
\item Those principles are meant to not just be extensionally adequate.

\item They are meant to explain why agents lose knowledge when considering some sets of options, like in the Red-Blue game.

\item In some sense, they are meant to be part of reductive explanations.

\end{itemize}
\end{frame}

\begin{frame}

\frametitle{Inputs to the Explanation}
\label{inputstotheexplanation}

These reductive explanations take as primitive inputs

\begin{itemize}
\item Evidential Probability

\item Evidence

\end{itemize}
I'm not going to worry about evidential probability here, but I am going to worry a lot about evidence.

\end{frame}

\section{The Problems with Evidence}
\label{theproblemswithevidence}

\begin{frame}

\frametitle{The Red-Blue Game and Evidence}
\label{thered-bluegameandevidence}

Consider a version of the game where

\begin{itemize}
\item The red sentence is two plus two equals four.

\item The blue sentence is something that, if known, would be part of the agent's evidence.

\end{itemize}
Hypothesis:

\begin{itemize}
\item We can get situations where the only rational play is Red-True, but in ordinary circumstances, the agent would know the blue sentence is true, and it would be part of their evidence.

\end{itemize}
\end{frame}

\begin{frame}

\frametitle{An Example}
\label{anexample}

\begin{itemize}
\item I see someone, call them Rahul, across the room in a restaurant in Ann Arbor.

\item Rahul is someone I know well, and can recognise, but I had no idea he was in town.

\item Still, the ordinary situation is that I know Rahul is here.

\item Indeed, the ordinary situation is that Rahul being in this restaurant is part of my evidence.

\end{itemize}
Now play a version of the game with:

\begin{itemize}
\item \textcolor{darkred}{Two plus two equals four.}

\item \textcolor{darkblue}{Rahul is in this restaurant.}

\end{itemize}
\end{frame}

\begin{frame}

\frametitle{The Challenge}
\label{thechallenge}

\begin{itemize}
\item This doesn't threaten the extensional adequacy of the conditional principle.

\item This set of views is consistent: E=K, and I don't know Rahul is here, so it's not part of my evidence that Rahul is here, so the evidential probability of Rahul being in Ann Arbor is not high enough to choose Blue.

\item But this explanation is not a reductive explanation of why I don't know Rahul is here.

\item It reasons from the lack to knowledge to the lack of evidence, and I want an explanation that goes the other way around.

\end{itemize}
\end{frame}

\begin{frame}

\frametitle{Some Ways Out}
\label{somewaysout}

\begin{enumerate}
\item Insist that evidence is only ever phenomenological, and the red-blue game never defeats phenomenological knowledge.

\item Give up on the project of providing reductive explanations for why changing practical circumstances lead to loss of knowledge.

\end{enumerate}
Neither seems particularly plausible.

\end{frame}

\begin{frame}

\frametitle{Multiple Solutions}
\label{multiplesolutions}

One cost of the explanation being non-reductive is that the following position is also consistent:

\begin{itemize}
\item E=K

\item Agents loses knowledge that \emph{p} when the evidential probability of \emph{p} is not close enough to one.

\item Since \emph{p} is part of my evidence, its evidential probability is 1, so it is close enough to 1.

\item So there is no threat from pragmatic encroachment to knowledge here.

\end{itemize}
A non-reductive account of when pragmatic effects matter is, in this case, a non-predictive account.

\end{frame}

\section{A Solution}
\label{asolution}

\begin{frame}

\frametitle{The Old Solution in Symbols}
\label{theoldsolutioninsymbols}

Let \emph{K} be the agent's evidence, and \emph{A, B} be relevant choices, \emph{E} the expected value function, and \emph{p} something the agent may or may not know. Then the agent knows \emph{p} only if:


\begin{equation*}
E(A | K) \geq E(B | K) \leftrightarrow E(A | K \wedge p) \geq E(B | K \wedge p)
\end{equation*}
\end{frame}

\begin{frame}

\frametitle{Towards a New Solution}
\label{towardsanewsolution}

\begin{itemize}
\item The problem is that we don't know whether $K$ includes $p$ or not.

\item So let $K$ now include only the uncontroversial part of the agent's evidence. So possibly the evidence is $K \wedge p$, possibly it is $K$.

\end{itemize}
\end{frame}

\begin{frame}

\frametitle{Split the Difference}
\label{splitthedifference}

For any action \emph{X}, define a new function \emph{V} as follows.

\begin{quote}

$V(X) = \frac{E(X | K) + E(X | K \wedge p)}{2}$
\end{quote}
It's the average of the expected values of $X$ with and without $p$ in the evidence. 

\end{frame}

\begin{frame}

\frametitle{The New Pragmatic Constraint}
\label{thenewpragmaticconstraint}

If \emph{p} is something that might be known, and is part of evidence if known, then the pragmatic constraint is that for any relevant \emph{A, B}:

\begin{quote}

$V(A) \geq V(B) \leftrightarrow E(A | p) \geq E(B | p)$
\end{quote}
If $A$ beats $B$ given $p$, then $A$ must also do better than $B$ on this `split the difference' criteria.

\end{frame}

\begin{frame}

\frametitle{Good Features of Solution}
\label{goodfeaturesofsolution}

\begin{itemize}
\item It gets the obvious cases (like Rahul in the restaurant), right.

\item It does not presuppose that we know what evidence the agent has in order to apply the rule.

\end{itemize}
\end{frame}

\begin{frame}

\frametitle{Bad Features of Solution}
\label{badfeaturesofsolution}

\begin{itemize}
\item It isn't obvious how it is going to generalise to cases where there are multiple propositions that might or might not be part of evidence. \pause

\item It looks completely ad hoc.

\end{itemize}
\end{frame}

\section{Gamifying the Problem}
\label{gamifyingtheproblem}

\begin{frame}

\frametitle{Newcomb's Problem as a Game}
\label{newcombsproblemasagame}

\begin{itemize}
\item It is interesting to think of some philosophical problems as games, especially when they involve interactions of rational agents.

\item Here, for example, is the game table for Newcomb's problem, with the familiar human as Row, and the demon as Column.

\end{itemize}

\begin{center}
\begin{tabular}{r | c c}
 & Predict 1 Box & Predict 2 Boxes \\ \hline
Choose 1 Box & 1000, 1 & 0,0 \\
Choose 2 Boxes & 1001, 0 & 1, 1
\end{tabular}
\end{center}
\pause


Note that the unique Nash equilibrium of the game is the bottom right corner.

\end{frame}

\begin{frame}

\frametitle{The Interpretation Game}
\label{theinterpretationgame}

There are two players:

\begin{enumerate}
\item Human

\item The Radical Interpreter

\end{enumerate}
Here are their goals:

\begin{itemize}
\item The Radical Interpreter assigns mental states (including evidence) to human in such a way as to correctly predict human's actions (assuming human is rational).

\item Human acts so as to maximise evidential expected utility, where the evidence is what the radical interpreter says the evidence is.

\end{itemize}
\end{frame}

\begin{frame}

\frametitle{A Version of the Game}
\label{aversionofthegame}

\begin{itemize}
\item Human faces a choice between taking and declining a bet on \emph{p}.

\item If bet wins, it wins 1 util, if it loses, it loses 100 utils.

\item \emph{p} is like the claim that Rahul is in the restaurant; it is unclear whether it is in human's evidence.

\item If $K$ is the rest of human's evidence, then $\Pr(p | K) = 0.9$.

\item The Radical Interpreter has to choose whether \emph{p} is part of the evidence or not.

\item Human has to decide whether to take the bet or not.

\item The Radical Interpreter gets what they want if human takes the bet iff \emph{p} is part of their evidence.

\end{itemize}
\end{frame}

\begin{frame}

\frametitle{Table for the Game}
\label{tableforthegame}


\begin{center}
\begin{tabular}{r | c c}
& $p \in E$ & $p \notin E$ \\ \hline
Take the bet & 1, 1 & -9.1, 0 \\
Decline the bet & 0, 0 & 0, 1
\end{tabular}
\end{center}


\begin{itemize}
\item Since the bet is rational iff \emph{p} is part of evidence, The Radical Interpreter wins in the top-left and lower-right quadrants, and loses otherwise.

\item In the bottom row, human gets a payout of 0, since the bet is declined.

\item In the top-right, the bet is a sure winner, so it's expected return is 1.

\item In the top-left, bet wins with probability 0.9, so its expected payout is --9.1.

\end{itemize}
\end{frame}

\begin{frame}

\frametitle{Equilibria of the Game}
\label{equilibriaofthegame}

This is a coordination game, and like most coordination games, it has multiple Nash equilibria.


\begin{center}
\begin{tabular}{r | c c}
& $p \in E$ & $p \notin E$ \\ \hline
Take the bet & \textbf{1, 1} & -9.1, 0 \\
Decline the bet & 0, 0 & \textbf{0, 1}
\end{tabular}
\end{center}


That corresponds to the conditional principle not setting a unique solution to what the agent's evidence\slash knowledge is.

\end{frame}

\section{Solving Coordination Games}
\label{solvingcoordinationgames}

\begin{frame}

\frametitle{Stag Hunt}
\label{staghunt}


\begin{center}
\begin{tabular}{r | c c}
& $a$ & $b$  \\\hline
$A$ & 5, 5 & 0, 4 \\
$B$ & 4, 0 & 2, 2
\end{tabular}
\end{center}


\begin{itemize}
\item This game has two equilibria, $Aa$ and $Bb$.

\item Let's talk about the choice between them.

\end{itemize}
\end{frame}

\begin{frame}

\frametitle{Pareto-Dominant}
\label{pareto-dominant}

\begin{itemize}
\item The $Aa$ equilibrium is better for both players than the $Bb$ equilibrium.

\item That is, it is \textbf{Pareto-dominant}.

\item Some theorists think we should select Pareto-dominant equilibria, when they are available.

\end{itemize}
\end{frame}

\begin{frame}

\frametitle{Risk-Dominant}
\label{risk-dominant}

\begin{itemize}
\item Each player does best playing $Bb$ if they think it is 50\slash 50 which equilibrium strategy the other player will play.

\item That is (simplifying a little), the $Bb$ strategy is \textbf{risk-dominant}.

\item Some other theorists think we should select risk-dominant equilibria, when they are available.

\end{itemize}
\end{frame}

\begin{frame}

\frametitle{Looking Ahead}
\label{lookingahead}

Here's the quick version of the rest of the paper.

\begin{enumerate}
\item In the game between Human and The Radical Interpreter, there is a Pareto-dominant and a Risk-dominant equilibria.

\item Risk dominance is a better equilibrium choice rule than Pareto-dominance.

\item The risk-dominant equilibria is the low evidence equilibria.

\item So that's what The Radical Interpreter will choose.

\item So in that case Human does not have \emph{p} in their evidence.

\end{enumerate}
\end{frame}

\begin{frame}

\frametitle{The Two Equilibria}
\label{thetwoequilibria}


\begin{center}
\begin{tabular}{r | c c}
& $p \in E$ & $p \notin E$ \\ \hline
Take the bet & 1, 1 & -9.1, 0 \\
Decline the bet & 0, 0 & 0, 1
\end{tabular}
\end{center}


\begin{itemize}
\item There is literally nothing to choose between them for The Radical Interpreter.

\item Human would prefer the top-left equilibrium.

\item But it is very risky; the lower-right equilibrium is safer.

\end{itemize}
\end{frame}

\begin{frame}

\frametitle{Why Prefer Risk-Dominance}
\label{whypreferrisk-dominance}

\begin{itemize}
\item It does better in games where people don't know exactly what the payoffs are.

\item In epistemic games, the Human (at least) doesn't know exactly what the payoffs are.

\end{itemize}
\end{frame}

\begin{frame}

\frametitle{An Example}
\label{anexample}

Assume that Row and Column are playing a version of this game, with for now unknown $x$.


\begin{center}
\begin{tabular}{r | c c}
& $a$ & $b$ \\ \hline
$A$ & 4, 4 & 0, $x$ \\
$B$ & $x$, 0 & $x$, $x$
\end{tabular}
\end{center}


\begin{itemize}
\item $x$ will be chosen at random from $[-1, 5]$.

\item Column will be told the value for $x$

\item Row will be told $x$ with a small error margin, chosen at random from $[-\varepsilon, \varepsilon]$.

\end{itemize}
\end{frame}

\begin{frame}

\frametitle{Solving this Global Game}
\label{solvingthisglobalgame}

\begin{itemize}
\item There is only one solution (more or less) to this game.

\item Both players play $B$ if they get a `signal' of more than 2.

\item Both players play $A$ if they get a `signal' of less than 2.

\item Strictly speaking, we don't rule out a whole bunch of options for what to do when the signal is 2.

\item We only need to use iterated deletion of strongly dominated strategies to solve this game.

\end{itemize}
\end{frame}

\begin{frame}

\frametitle{Generalising}
\label{generalising}

\begin{itemize}
\item Any coordination game can be easily embedded in a global game like this.

\item One of the payoffs is replaced with an unknown variable, and the player gets a noisy signal of the payoff.

\item In general, the solution to the most natural form of the global game is to play the risk-dominant equilibria.

\item And that's a realistic setting for what Human faces.

\end{itemize}
\end{frame}

\section{Odds and Ends}
\label{oddsandends}

\begin{frame}

\frametitle{Progress Made}
\label{progressmade}

\begin{itemize}
\item We have a thing to say about cases like Rahul in the restaurant.

\item It doesn't rely on figuring out antecedently what agent's evidence is.

\item It is consistent with the story that agents lose knowledge that \emph{p} when they can't conditionalise on \emph{p}.

\end{itemize}
\end{frame}

\begin{frame}

\frametitle{Downsides}
\label{downsides}

\begin{itemize}
\item Much more complicated.

\item It turns out to make a difference whether \emph{p} is possible evidence or not.

\end{itemize}
\end{frame}

\mode<all>
\end{document}

\end{document}\mode*
